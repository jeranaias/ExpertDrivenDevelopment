% =============================================================================
% SOP: Expert-Driven AI-Assisted Development -- Version 5.0
% Compiled with pdflatex
% =============================================================================
\documentclass[11pt,letterpaper]{article}

% ---------------------------------------------------------------------------
% Packages
% ---------------------------------------------------------------------------
\usepackage[margin=1in]{geometry}
\usepackage{fancyhdr}
\usepackage{titlesec}
\usepackage[colorlinks=true,linkcolor=black,urlcolor=blue,bookmarks=true]{hyperref}
\usepackage{xcolor}
\usepackage{booktabs}
\usepackage{longtable}
\usepackage{enumitem}
\usepackage{tocloft}
\usepackage{lastpage}
\usepackage{parskip}

% ---------------------------------------------------------------------------
% Color definitions
% ---------------------------------------------------------------------------
\definecolor{scarlet}{RGB}{180,30,30}
\definecolor{darkgray}{RGB}{60,60,60}

% ---------------------------------------------------------------------------
% Header / Footer
% ---------------------------------------------------------------------------
\pagestyle{fancy}
\fancyhf{}
\fancyhead[L]{\small\textsc{SOP: Expert-Driven AI-Assisted Development}}
\fancyhead[R]{\small\textbf{UNCLASSIFIED}}
\fancyfoot[L]{\small Version 5.0}
\fancyfoot[R]{\small Page \thepage\ of \pageref{LastPage}}
\renewcommand{\headrulewidth}{0.4pt}
\renewcommand{\footrulewidth}{0.4pt}

% ---------------------------------------------------------------------------
% Section formatting
% ---------------------------------------------------------------------------
\titleformat{\section}
  {\normalfont\Large\bfseries\color{scarlet}}
  {\thesection.}{0.5em}{}
\titleformat{\subsection}
  {\normalfont\large\bfseries\color{darkgray}}
  {\thesubsection}{0.5em}{}
\titleformat{\subsubsection}
  {\normalfont\normalsize\bfseries}
  {\thesubsubsection}{0.5em}{}

% ---------------------------------------------------------------------------
% Table of contents formatting
% ---------------------------------------------------------------------------
\renewcommand{\cftsecleader}{\cftdotfill{\cftdotsep}}
\setcounter{tocdepth}{2}

% ---------------------------------------------------------------------------
% List formatting
% ---------------------------------------------------------------------------
\setlist[itemize]{nosep,left=0pt .. 1.5em}
\setlist[enumerate]{nosep,left=0pt .. 1.5em}

% ---------------------------------------------------------------------------
% Callout box command
% ---------------------------------------------------------------------------
\newcommand{\calloutbox}[2]{%
  \medskip
  \noindent\fbox{\parbox{\dimexpr\textwidth-2\fboxsep-2\fboxrule}{%
    \textbf{#1}\par\smallskip #2}}%
  \medskip
}

% ---------------------------------------------------------------------------
% Blockquote environment
% ---------------------------------------------------------------------------
\newenvironment{sopblockquote}{%
  \begin{quote}\itshape
}{%
  \end{quote}
}

% =============================================================================
\begin{document}

% =============================================================================
% TITLE PAGE
% =============================================================================
\thispagestyle{empty}

\begin{center}

\vspace*{1.5in}

{\Large\textbf{UNCLASSIFIED // Distribution Unlimited}}

\vspace{0.75in}

{\LARGE\textbf{Standard Operating Procedure}}

\vspace{0.3in}

{\Huge\bfseries\color{scarlet} Expert-Driven AI-Assisted Development}

\vspace{0.3in}

{\Large A Methodology for Building Institutional Capability}

\vspace{1in}

\begin{tabular}{rl}
\textbf{Document Version:} & 5.0 \\[4pt]
\textbf{Effective Date:}    & February 2026 \\[4pt]
\textbf{Classification:}    & UNCLASSIFIED // Distribution Unlimited \\[4pt]
\textbf{Prepared by:}       & SSgt Jesse C. Morgan \\
                             & Marine Corps Detachment Presidio of Monterey \\
                             & Defense Language Institute Foreign Language Center \\
\end{tabular}

\vfill

\rule{0.8\textwidth}{0.4pt}

\medskip

{\small\itshape Core Principle: Use AI to build things that don't need AI.\\
The tools last. The AI was just how we built them.}

\medskip

\rule{0.8\textwidth}{0.4pt}

\vspace{0.5in}

{\small\textbf{UNCLASSIFIED}}

\end{center}

\newpage

% =============================================================================
% TABLE OF CONTENTS
% =============================================================================
\tableofcontents
\newpage

% =============================================================================
% SECTION 1: PURPOSE AND SCOPE
% =============================================================================
\section{Purpose and Scope}
\label{sec:purpose-and-scope}

\subsection{Purpose}
\label{sec:purpose}

This Standard Operating Procedure establishes the framework, governance, and workflow for
AI-assisted tool development by subject-matter experts within Department of War (DoW) organizations.
Expert-Driven Development enables domain experts to build, document, and sustain institutional
tools using AI as a development accelerator---without requiring a background in software engineering,
contracting, or formal IT acquisition.

\subsection{EDD IS / EDD ISN'T}
\label{sec:is-isnt}

\begin{minipage}[t]{0.47\textwidth}
\subsubsection*{EDD IS}
\begin{itemize}
  \item A framework for building institutional tools using AI
  \item A way to turn subject-matter expertise into working software
  \item A governance model for responsible AI-assisted development
  \item A training pipeline that produces AI-fluent builders
  \item A documentation standard that ensures tool continuity
  \item Compatible with existing DoW cybersecurity and privacy requirements
\end{itemize}
\end{minipage}%
\hfill
\begin{minipage}[t]{0.47\textwidth}
\subsubsection*{EDD ISN'T}
\begin{itemize}
  \item A replacement for professional software development or IT acquisition
  \item An authorization to bypass cybersecurity or privacy controls
  \item A way to build systems that handle classified data
  \item A shortcut that eliminates the need for documentation
  \item A single AI tool or platform
  \item Limited to any specific programming language or technology stack
\end{itemize}
\end{minipage}

\subsection{Scope}
\label{sec:scope}

This SOP covers the complete lifecycle of AI-assisted tool development: from problem identification
and compliance review through prototyping, testing, documentation, deployment, and sustainment.
It applies to tools built on approved platforms using approved AI assistants that handle
unclassified data only.

\subsection{Applicability}
\label{sec:applicability}

This SOP applies to all personnel who develop, review, supervise, or maintain tools built
using the Expert-Driven Development framework. It is designed for adoption at the unit level
and can be tailored to organization-specific requirements while maintaining compliance with
DoW-wide cybersecurity and privacy directives.

% =============================================================================
% SECTION 2: THE FOUR-LAYER FRAMEWORK
% =============================================================================
\section{The Four-Layer Framework}
\label{sec:four-layer-framework}

Expert-Driven Development operates through four interdependent layers. Each layer builds
on the previous one, and all four must be present for the framework to produce sustainable results.

\subsection{Layer 1: Direct Tool Development}
\label{sec:layer-1}

Subject-matter experts use AI to build tools that solve real problems in their organizations.
The expert provides domain knowledge---the understanding of the problem, the workflow, the
users, and the requirements. AI provides the development capability---translating that
domain knowledge into working code, interfaces, and automation.

\calloutbox{Key Principle}{%
The expert drives the development. AI is the accelerator, not the decision-maker.
Every design choice, workflow decision, and feature prioritization comes from the
person who understands the problem.}

\subsection{Layer 2: Process Liberation}
\label{sec:layer-2}

By removing the dependency on contractors, acquisition timelines, and IT development queues,
EDD liberates the development process. Personnel can identify a problem and begin building
a solution within the same week---not the same fiscal year. This layer addresses the
structural bottleneck that prevents organizations from responding to their own needs.

\subsection{Layer 3: Capability Cultivation}
\label{sec:layer-3}

Each build develops the builder. The five-course training curriculum and the 201-level
skills framework cultivate six core competencies: Context Assembly, Quality Judgment,
Task Decomposition, Iterative Refinement, Workflow Integration, and Frontier Recognition.
Personnel who complete EDD projects gain transferable skills in AI-assisted development,
systems thinking, user-centered design, and technical documentation. These capabilities
persist after the specific tool is deployed and compound across subsequent projects.

\subsection{Layer 4: Documentation and Replication}
\label{sec:layer-4}

Every tool built under EDD is documented to a standard that enables another person to
understand, maintain, modify, and replicate the tool without the original builder present.
This layer ensures that institutional knowledge does not walk out the door when personnel
transfer. Documentation is not an afterthought---it is a core deliverable of every project.

% =============================================================================
% SECTION 3: THE CREATOR MINDSET
% =============================================================================
\section{The Creator Mindset}
\label{sec:creator-mindset}

EDD requires a fundamental shift in how personnel interact with AI tools. Most users
operate as consumers---they use AI to get answers, summarize text, or draft emails.
EDD trains personnel to operate as creators---using AI to build things that other
people use.

\subsection{Consumer vs Creator}
\label{sec:consumer-vs-creator}

\begin{longtable}{@{}p{3cm}p{5cm}p{5.5cm}@{}}
\toprule
\textbf{Dimension} & \textbf{Consumer} & \textbf{Creator} \\
\midrule
\endfirsthead
\toprule
\textbf{Dimension} & \textbf{Consumer} & \textbf{Creator} \\
\midrule
\endhead
\bottomrule
\endlastfoot
\textbf{Output} & Answers for yourself & Tools for others \\
\midrule
\textbf{Interaction} & Single prompt, single response & Iterative conversation toward a build \\
\midrule
\textbf{Scope} & Personal productivity & Institutional capability \\
\midrule
\textbf{Duration} & One-time use & Persistent, maintained artifact \\
\midrule
\textbf{Accountability} & None---private use & Documented, reviewed, and sustained \\
\midrule
\textbf{Knowledge transfer} & None & Documentation package enables replication \\
\end{longtable}

\subsection{The Artifact Test}
\label{sec:artifact-test}

The simplest way to determine whether someone is operating as a consumer or a creator:

\begin{sopblockquote}
\textbf{Does your AI interaction produce an artifact that someone else can use?}

If the answer is yes---a tool, an application, a system, a workflow---you are creating.
If the answer is no---a summary, an email draft, a personal answer---you are consuming.
Both are valid. EDD focuses on creating.
\end{sopblockquote}

% =============================================================================
% SECTION 4: COMPLIANCE AND SECURITY
% =============================================================================
\section{Compliance and Security}
\label{sec:compliance-and-security}

All development under this SOP must comply with existing DoW cybersecurity, privacy,
OPSEC, and records management requirements. AI-assisted development does not create
new exceptions to these requirements---it operates within them.

\subsection{Data Classification}
\label{sec:data-classification}

EDD tools handle \textbf{unclassified data only}. The following data types are
prohibited from use in AI-assisted development:

\calloutbox{Prohibited Data}{%
\begin{itemize}
  \item Classified information (any level)
  \item Controlled Unclassified Information (CUI) unless the platform is CUI-authorized
  \item Personally Identifiable Information (PII) unless a Privacy Impact Assessment has been completed
  \item Protected Health Information (PHI)
  \item For Official Use Only (FOUO) data in non-authorized environments
  \item Operational plans, force movement data, or intelligence products
\end{itemize}}

\subsection{Privacy Requirements}
\label{sec:privacy-requirements}

Tools that collect, store, process, or display information about individuals must comply
with DoW privacy directives. The following apply:

\subsubsection{Privacy Impact Assessment (PIA)}

Required for any tool that collects or maintains PII. A PIA threshold analysis must be
completed during the Compliance Review phase (Phase 2) to determine whether a full PIA
is required.

\subsubsection{System of Records Notice (SORN)}

If the tool creates a new system of records retrievable by a personal identifier, a SORN
must be published in the Federal Register before the tool goes into production use.

\subsubsection{Appeals and Redress}

Any tool that makes decisions affecting individuals must include a mechanism for
individuals to review, contest, and correct information about themselves.

\subsection{Cybersecurity}
\label{sec:cybersecurity}

\subsubsection{Approved AI Tools}
\label{sec:approved-ai-tools}

The following AI tools are approved for use in EDD development. Authorization levels
indicate the scope of approved use:

\begin{longtable}{@{}p{3cm}p{4cm}p{6.5cm}@{}}
\toprule
\textbf{Tool} & \textbf{Authorization Level} & \textbf{Notes} \\
\midrule
\endfirsthead
\toprule
\textbf{Tool} & \textbf{Authorization Level} & \textbf{Notes} \\
\midrule
\endhead
\bottomrule
\endlastfoot
M365 Copilot & Enterprise-authorized & Available through existing DoW M365 licensing; data stays within tenant \\
\midrule
Azure OpenAI & Enterprise-authorized & Requires Azure Government subscription; FedRAMP High authorized \\
\midrule
Copilot Studio & Enterprise-authorized & Low-code bot builder within M365 ecosystem \\
\midrule
Gemini & Approved with restrictions & Unclassified use only; do not input PII or CUI \\
\midrule
Claude & Approved with restrictions & Unclassified use only; do not input PII or CUI \\
\midrule
C3 AI & Enterprise-authorized & DoW Enterprise AI platform; available through enterprise agreement \\
\end{longtable}

\subsubsection{Approved Platforms}
\label{sec:approved-platforms}

Tools built under EDD must be deployed on approved platforms:

\begin{longtable}{@{}p{3.5cm}p{5cm}p{5cm}@{}}
\toprule
\textbf{Platform} & \textbf{Use Case} & \textbf{Authorization} \\
\midrule
\endfirsthead
\toprule
\textbf{Platform} & \textbf{Use Case} & \textbf{Authorization} \\
\midrule
\endhead
\bottomrule
\endlastfoot
Microsoft Power Platform & Low-code applications, automation, dashboards & Enterprise-authorized within DoW M365 \\
\midrule
SharePoint Online & Web-based tools, document management, portals & Enterprise-authorized within DoW M365 \\
\midrule
Azure Government & Cloud-hosted applications, APIs, databases & FedRAMP High; requires subscription \\
\midrule
GitHub (public) & Open-source projects, static sites, documentation & Unclassified public release only \\
\end{longtable}

\subsubsection{When an ATO Is Required}
\label{sec:ato-requirements}

An Authority to Operate (ATO) is required when a tool:

\begin{itemize}
  \item Connects to or operates on the DoW Information Network (DoWIN)
  \item Processes, stores, or transmits CUI
  \item Integrates with other authorized information systems
  \item Operates outside the inherited authorization boundary of its host platform
\end{itemize}

Tools built entirely within the inherited authorization boundary of an approved platform
(e.g., a Power App within DoW M365) typically do not require a separate ATO but must
still comply with the platform's security controls.

\subsubsection{Security Controls}
\label{sec:security-controls}

All tools must implement security controls appropriate to their data sensitivity and
deployment environment, including:

\begin{itemize}
  \item Authentication via CAC or enterprise identity provider
  \item Role-based access control (RBAC) where applicable
  \item Audit logging of user actions and data access
  \item Input validation and output encoding
  \item Encryption of data in transit (TLS 1.2+) and at rest
\end{itemize}

\subsection{OPSEC}
\label{sec:opsec}

Developers must apply OPSEC principles throughout the development lifecycle. Do not
include operational details, unit-specific procedures, force structure information,
or vulnerability assessments in AI prompts, tool documentation, or publicly accessible
code repositories. When in doubt, consult your organization's OPSEC officer before
proceeding.

\subsection{Records Management}
\label{sec:records-management}

Development artifacts (documentation packages, source code, deployment records) are
official records and must be managed in accordance with DoW records management directives.
Retain all project documentation for the duration specified by the applicable records
schedule. The documentation package produced in Phase 5 serves as the primary record
of each development effort.

\subsection{Incident Response}
\label{sec:incident-response}

If a security incident occurs involving an EDD tool (data breach, unauthorized access,
malware, or compromise of an AI tool), personnel must:

\begin{enumerate}
  \item Immediately cease use of the affected tool
  \item Report the incident to the organization's Cybersecurity Office within 1 hour
  \item Preserve all logs, artifacts, and evidence related to the incident
  \item Notify the chain of command and the EDD Program Coordinator
  \item Cooperate with the incident response investigation
  \item Do not attempt to remediate the vulnerability without authorization
\end{enumerate}

% =============================================================================
% SECTION 5: DEVELOPMENT PREREQUISITES
% =============================================================================
\section{Development Prerequisites}
\label{sec:development-prerequisites}

\subsection{Required}
\label{sec:required}

\begin{itemize}
  \item Completed EDD training (minimum: AI Fluency Fundamentals)
  \item Supervisor approval for the specific development project
  \item Access to at least one approved AI tool
  \item Access to at least one approved deployment platform
  \item A defined problem that meets the criteria in Phase 1
  \item Completed Problem Definition Template
\end{itemize}

\subsection{Not Required}
\label{sec:not-required}

\begin{itemize}
  \item A background in software engineering or computer science
  \item Prior programming experience
  \item A contract, funding, or acquisition document
  \item An IT service request or development ticket
  \item Permission from a centralized IT authority (unless ATO is required)
\end{itemize}

\subsection{Obtaining Access}
\label{sec:obtaining-access}

\subsubsection{Microsoft 365}

Most DoW personnel already have M365 accounts. Power Platform access may require
a license assignment through your organization's M365 administrator. Contact your
local IT support to verify your license includes Power Apps, Power Automate, and
Copilot Studio.

\subsubsection{AI Tools}

The following table provides recommendations for obtaining access to approved AI tools:

\begin{longtable}{@{}p{2.5cm}p{5.5cm}p{5.5cm}@{}}
\toprule
\textbf{Tool} & \textbf{How to Obtain Access} & \textbf{Recommendation} \\
\midrule
\endfirsthead
\toprule
\textbf{Tool} & \textbf{How to Obtain Access} & \textbf{Recommendation} \\
\midrule
\endhead
\bottomrule
\endlastfoot
M365 Copilot & Enterprise license assignment through M365 admin & Start here if your organization has Copilot licenses \\
\midrule
Azure OpenAI & Azure Government subscription; request through cloud team & Best for custom API-driven development \\
\midrule
Copilot Studio & Included with Power Platform licensing & Best for building conversational AI assistants \\
\midrule
Gemini & Web access at gemini.google.com (unclassified only) & Good general-purpose coding assistant \\
\midrule
Claude & Web access at claude.ai (unclassified only) & Strong for complex reasoning and documentation \\
\midrule
C3 AI & Enterprise agreement; contact your AI/ML program office & Best for data-heavy enterprise applications \\
\end{longtable}

% =============================================================================
% SECTION 6: DEVELOPMENT WORKFLOW
% =============================================================================
\section{Development Workflow}
\label{sec:development-workflow}

The EDD development workflow consists of eight phases. Each phase has defined inputs,
outputs, and time estimates. Phases are sequential, but iteration between adjacent
phases is expected and encouraged.

% --- Phase 1 ---
\subsection{Phase 1: Problem Definition}
\label{sec:phase-1}

\noindent\textit{Estimated time: 2--4 hours}

Identify and document the problem you intend to solve. Complete the Problem
Definition Template. Define the current state, desired state, affected users,
and success criteria. A well-defined problem is the single strongest predictor
of a successful build.

\textbf{Output:} Completed Problem Definition Template.

% --- Phase 2 ---
\subsection{Phase 2: Compliance Review}
\label{sec:phase-2}

\noindent\textit{Estimated time: 1--4 hours}

Review the proposed tool against compliance requirements. Complete the Compliance
Checklist. Determine data classification, privacy requirements, and whether an
ATO is needed. Identify the appropriate deployment platform. Obtain supervisor
approval to proceed.

\textbf{Output:} Completed Compliance Checklist with supervisor signature.

% --- Phase 3 ---
\subsection{Phase 3: Rapid Prototyping}
\label{sec:phase-3}

\noindent\textit{Estimated time: 8--20 hours}

Build the tool using AI-assisted development. Start with the minimum viable
feature set identified in Phase 1. Use iterative prompting to develop, test,
and refine functionality. Maintain a development journal documenting key
decisions, prompts that worked, and problems encountered.

\textbf{Output:} Working prototype with core functionality.

% --- Phase 4 ---
\subsection{Phase 4: User Testing}
\label{sec:phase-4}

\noindent\textit{Estimated time: 4--8 hours}

Put the prototype in front of actual users. Observe how they interact with it.
Collect feedback on usability, functionality, and missing features. Iterate on
the prototype based on user feedback. A minimum of three users should test the
tool before proceeding to documentation.

\textbf{Output:} User-tested prototype with documented feedback and revisions.

% --- Phase 5 ---
\subsection{Phase 5: Documentation}
\label{sec:phase-5}

\noindent\textit{Estimated time: 20--40 hours (first project) / 8--16 hours (subsequent)}

Produce the documentation package. This is the most critical phase for institutional
value. The documentation must enable someone who has never seen the tool to understand
what it does, how it works, how to maintain it, how to modify it, and how to
replicate it. First-time builders should expect to spend significantly more time
on documentation; the investment decreases with experience.

\textbf{Output:} Complete documentation package.

% --- Phase 6 ---
\subsection{Phase 6: QA Review}
\label{sec:phase-6}

\noindent\textit{Estimated time: 2--4 hours}

A designated QA Reviewer examines the tool and its documentation against the
QA Checklist. The reviewer verifies functionality, security compliance,
documentation completeness, and readiness for deployment. The reviewer must
not be the same person who built the tool.

\textbf{Output:} Completed QA Checklist with reviewer sign-off or remediation requirements.

% --- Phase 7 ---
\subsection{Phase 7: Deployment}
\label{sec:phase-7}

\noindent\textit{Estimated time: 1--2 hours}

Deploy the tool to its production environment. Register the tool in the Tool
Registry. Communicate availability to intended users. Establish the sustainment
plan, including who is responsible for ongoing maintenance and how issues will
be reported and resolved.

\textbf{Output:} Deployed tool, completed Tool Registry entry, and sustainment plan.

% --- Phase 8 ---
\subsection{Phase 8: Sustainment}
\label{sec:phase-8}

\noindent\textit{Estimated time: Ongoing}

Maintain the tool in its operational environment. Monitor usage metrics, address
user-reported issues, apply updates as needed, and update documentation to reflect
changes. Plan for turnover: when the original builder departs, the documentation
package must be sufficient for a successor to assume maintenance responsibility.

\textbf{Output:} Operational tool with current documentation and identified maintainer.

% =============================================================================
% SECTION 7: ROLES AND RESPONSIBILITIES
% =============================================================================
\section{Roles and Responsibilities}
\label{sec:roles-and-responsibilities}

\subsection{Developer}
\label{sec:role-developer}

The individual who builds the tool. Responsible for completing all eight phases of the
development workflow, maintaining compliance with this SOP, producing the documentation
package, and sustaining the tool through its operational lifecycle. Any personnel member
who has completed the required training and received supervisor approval may serve as a
developer.

\subsection{QA Reviewer}
\label{sec:role-qa-reviewer}

The individual who reviews the tool and documentation in Phase 6. The QA Reviewer must
be a different person from the developer. Designation criteria:

\begin{itemize}
  \item Has completed EDD training (minimum: AI Fluency Fundamentals)
  \item Has built at least one tool under EDD, or has been designated by the Program Coordinator
  \item Is familiar with the compliance requirements in Section~\ref{sec:compliance-and-security}
  \item Has no conflict of interest with the tool being reviewed
\end{itemize}

\subsection{Supervisor}
\label{sec:role-supervisor}

The developer's direct supervisor. Responsible for approving the development project,
allocating duty time for development, and ensuring the tool meets organizational needs.
Supervisors are required to complete the Supervisor Orientation before approving their
first EDD project. This orientation ensures supervisors understand the framework, can
evaluate proposals, and know what good AI-assisted output looks like.

\subsection{Program Coordinator}
\label{sec:role-coordinator}

The individual responsible for managing the EDD program at the organizational level.
Maintains the Tool Registry, designates QA Reviewers, coordinates training, tracks
program metrics, and reports program status to leadership. The Program Coordinator
may also serve as a developer or QA Reviewer.

\subsection{Cybersecurity Office}
\label{sec:role-cybersecurity}

The organization's cybersecurity team. Provides guidance on security controls, reviews
ATO requirements, responds to security incidents involving EDD tools, and validates
that tools comply with cybersecurity directives. Consulted during Phase 2 (Compliance
Review) for any tool that connects to the network or processes sensitive data.

\subsection{Privacy Officer}
\label{sec:role-privacy}

The organization's designated privacy official. Reviews PIA threshold analyses,
advises on SORN requirements, and ensures tools that handle PII comply with DoW
privacy directives. Consulted during Phase 2 for any tool that collects, stores,
or displays information about individuals.

% =============================================================================
% SECTION 8: TRAINING REQUIREMENTS
% =============================================================================
\section{Training Requirements}
\label{sec:training-requirements}

The v5 curriculum comprises five courses organized by audience and prerequisite chain.

\subsection{Required}
\label{sec:training-required}

The following course is the universal prerequisite and must be completed before beginning any EDD activity:

\begin{longtable}{@{}p{4cm}p{3cm}p{2cm}p{4.5cm}@{}}
\toprule
\textbf{Course} & \textbf{Audience} & \textbf{Duration} & \textbf{Prerequisite} \\
\midrule
\endfirsthead
\toprule
\textbf{Course} & \textbf{Audience} & \textbf{Duration} & \textbf{Prerequisite} \\
\midrule
\endhead
\bottomrule
\endlastfoot
AI Fluency Fundamentals & All personnel & 2 hours & None (universal prerequisite) \\
\end{longtable}

\subsection{Recommended for Builders}
\label{sec:training-recommended}

The following courses are recommended/elective for personnel who will build tools:

\begin{longtable}{@{}p{3.5cm}p{3.5cm}p{2cm}p{4.5cm}@{}}
\toprule
\textbf{Course} & \textbf{Audience} & \textbf{Duration} & \textbf{Prerequisite} \\
\midrule
\endfirsthead
\toprule
\textbf{Course} & \textbf{Audience} & \textbf{Duration} & \textbf{Prerequisite} \\
\midrule
\endhead
\bottomrule
\endlastfoot
Builder Orientation & Aspiring builders & 2 hours & AI Fluency Fundamentals \\
\midrule
Platform Training & Builders & 4 hours & Builder Orientation \\
\midrule
Advanced Workshop & Experienced builders & 4 hours & At least one deployed tool \\
\end{longtable}

\subsection{Recommended for Leadership}
\label{sec:training-leadership}

The following course is recommended for supervisors and leadership personnel:

\begin{longtable}{@{}p{3.5cm}p{4.5cm}p{2cm}p{3.5cm}@{}}
\toprule
\textbf{Course} & \textbf{Audience} & \textbf{Duration} & \textbf{Prerequisite} \\
\midrule
\endfirsthead
\toprule
\textbf{Course} & \textbf{Audience} & \textbf{Duration} & \textbf{Prerequisite} \\
\midrule
\endhead
\bottomrule
\endlastfoot
Supervisor Orientation & Supervisors approving EDD projects & 30 minutes & None \\
\end{longtable}

\subsection{Six 201-Level Skills}
\label{sec:six-skills}

The training curriculum develops six core competencies defined in the 201-level skills framework.
These skills represent the capabilities that separate effective AI-assisted builders from casual AI users:

\begin{enumerate}
  \item \textbf{Context Assembly}---The ability to gather and structure the right information so AI can produce useful output.
  \item \textbf{Quality Judgment}---The ability to evaluate AI-generated output for correctness, completeness, and fitness for purpose.
  \item \textbf{Task Decomposition}---The ability to break complex problems into discrete, AI-addressable tasks.
  \item \textbf{Iterative Refinement}---The ability to systematically improve AI output through structured feedback cycles.
  \item \textbf{Workflow Integration}---The ability to embed AI-assisted processes into existing organizational workflows.
  \item \textbf{Frontier Recognition}---The ability to identify where AI capability boundaries lie and adjust approach accordingly.
\end{enumerate}

\subsection{Instructor Certification}
\label{sec:instructor-certification}

Instructors for each course must meet the following certification requirements:

\begin{longtable}{@{}p{3.5cm}p{10cm}@{}}
\toprule
\textbf{Course} & \textbf{Requirements} \\
\midrule
\endfirsthead
\toprule
\textbf{Course} & \textbf{Requirements} \\
\midrule
\endhead
\bottomrule
\endlastfoot
AI Fluency Fundamentals & Completed course AND at least one builder course \\
\midrule
Builder Orientation & Completed all builder courses AND deployed at least one tool \\
\midrule
Platform Training & Completed all builder courses, deployed at least one tool, plus Power Platform proficiency \\
\midrule
Advanced Workshop & Completed all builder courses, deployed at least one tool, plus served as QA reviewer for 2+ tools \\
\midrule
Supervisor Orientation & Any qualified AI Fluency Fundamentals instructor \\
\end{longtable}

\textbf{Certification process:} Complete course as student $\rightarrow$ Build/deploy tool $\rightarrow$
Shadow instructor $\rightarrow$ Co-teach $\rightarrow$ Certification from Program Coordinator.

% =============================================================================
% SECTION 9: METRICS AND ASSESSMENT
% =============================================================================
\section{Metrics and Assessment}
\label{sec:metrics-and-assessment}

Metrics serve two purposes: demonstrating the value of individual tools and assessing
the health of the overall EDD program. All metrics should be collected consistently
and reported to leadership quarterly.

\subsection{Tool-Level Metrics}
\label{sec:tool-level-metrics}

Collected for each deployed tool:

\begin{longtable}{@{}p{3.5cm}p{4cm}p{6cm}@{}}
\toprule
\textbf{Metric} & \textbf{What It Measures} & \textbf{Collection Method} \\
\midrule
\endfirsthead
\toprule
\textbf{Metric} & \textbf{What It Measures} & \textbf{Collection Method} \\
\midrule
\endhead
\bottomrule
\endlastfoot
Active users & Adoption and utility & Platform analytics or usage logs \\
\midrule
Time saved per use & Efficiency gain & User survey or workflow comparison \\
\midrule
Error rate reduction & Quality improvement & Before/after error tracking \\
\midrule
Build time & Development efficiency & Developer-reported hours \\
\midrule
Development cost & Cost avoidance & Hours multiplied by composite labor rate \\
\midrule
User satisfaction & Quality of the solution & Post-deployment survey \\
\end{longtable}

\subsection{Program-Level Metrics}
\label{sec:program-level-metrics}

Collected across all EDD activity within the organization:

\begin{longtable}{@{}p{3.5cm}p{3.5cm}p{6.5cm}@{}}
\toprule
\textbf{Metric} & \textbf{What It Measures} & \textbf{Target} \\
\midrule
\endfirsthead
\toprule
\textbf{Metric} & \textbf{What It Measures} & \textbf{Target} \\
\midrule
\endhead
\bottomrule
\endlastfoot
Number of trained personnel & Pipeline capacity & Growing quarter-over-quarter \\
\midrule
Number of deployed tools & Program output & Growing quarter-over-quarter \\
\midrule
Completion rate (started vs deployed) & Process effectiveness & $>$70\% of started projects reach deployment \\
\midrule
Documentation compliance rate & Quality assurance & 100\% of deployed tools have complete packages \\
\midrule
Total estimated cost avoidance & Return on investment & Reported quarterly to leadership \\
\midrule
Security incidents & Risk management & Zero incidents per quarter \\
\end{longtable}

% =============================================================================
% SECTION 10: SCALING AND ADOPTION
% =============================================================================
\section{Scaling and Adoption}
\label{sec:scaling-and-adoption}

\subsection{90-Day Pilot Model}
\label{sec:pilot-model}

Organizations adopting EDD should begin with a 90-day pilot. The pilot provides a
controlled environment to validate the framework, train initial personnel, produce
the first tools, and establish organizational processes before expanding.

\begin{longtable}{@{}p{3cm}p{2.5cm}p{8cm}@{}}
\toprule
\textbf{Phase} & \textbf{Timeline} & \textbf{Activities} \\
\midrule
\endfirsthead
\toprule
\textbf{Phase} & \textbf{Timeline} & \textbf{Activities} \\
\midrule
\endhead
\bottomrule
\endlastfoot
Setup & Days 1--14 & Designate Program Coordinator, identify 2--4 pilot developers, conduct training, verify platform access \\
\midrule
First builds & Days 15--60 & Pilot developers complete Phases 1--5 on their first projects with coaching support \\
\midrule
QA and deployment & Days 61--75 & QA reviews, remediation, deployment, and Tool Registry entries \\
\midrule
Assessment & Days 76--90 & Collect metrics, brief leadership, decide on expansion \\
\end{longtable}

\subsection{Success / Failure Criteria}
\label{sec:success-failure-criteria}

\begin{minipage}[t]{0.47\textwidth}
\subsubsection*{Success Criteria}
\begin{itemize}
  \item At least 2 tools deployed and in active use
  \item All deployed tools have complete documentation packages
  \item Zero security incidents during the pilot
  \item Measurable time savings or error reduction demonstrated
  \item Leadership endorsement for expansion
\end{itemize}
\end{minipage}%
\hfill
\begin{minipage}[t]{0.47\textwidth}
\subsubsection*{Failure Indicators}
\begin{itemize}
  \item No tools reach deployment within 90 days
  \item Security or compliance violations occur
  \item Developers cannot obtain platform access
  \item Leadership does not allocate duty time for development
  \item Documentation requirements are waived or ignored
\end{itemize}
\end{minipage}

\bigskip

\subsection{Tool Registry}
\label{sec:tool-registry}

The Tool Registry is a centralized record of all tools built under EDD within the
organization. Every deployed tool must have an entry. The registry tracks:

\begin{itemize}
  \item Tool name and description
  \item Developer and current maintainer
  \item Deployment platform and URL
  \item Data classification
  \item User count and status (active, deprecated, retired)
  \item Documentation package location
  \item Date deployed and last updated
\end{itemize}

The Program Coordinator maintains the Tool Registry and reviews it monthly to identify
tools that need updates, lack a maintainer, or should be retired.

\subsection{Expansion Model}
\label{sec:expansion-model}

After a successful pilot, expansion follows a train-the-trainer model:

\begin{enumerate}
  \item Pilot developers become QA Reviewers and mentors for the next cohort
  \item Each cohort is 2--4 new developers with defined projects
  \item New cohorts begin every 60--90 days
  \item The Program Coordinator scales training, QA capacity, and registry management as the program grows
  \item Cross-organizational sharing is encouraged through the Tool Registry and documentation standards
\end{enumerate}

% =============================================================================
% SECTION 11: REFERENCES
% =============================================================================
\section{References}
\label{sec:references}

\subsection{Policy and Regulatory References}

\begin{itemize}
  \item DoW Directive 8140.01---Cyberspace Workforce Management
  \item DoW Instruction 5000.87---Operation of the Software Acquisition Pathway
  \item DoW Instruction 5200.48---Controlled Unclassified Information (CUI)
  \item DoW Instruction 5400.11---DoW Privacy and Civil Liberties Programs
  \item DoW Instruction 8500.01---Cybersecurity
  \item DoW Instruction 8510.01---Risk Management Framework (RMF) for DoW Systems
  \item DoW Instruction 8582.01---Security of Non-DoW Information Systems
  \item DoW Directive 3020.26---Department of War Continuity Policy
  \item DoWM 5200.01---DoW Information Security Program
  \item SECNAV M-5210.1---Department of the Navy Records Management Program
  \item NIST SP 800-53 Rev.\ 5---Security and Privacy Controls for Information Systems and Organizations
  \item OMB Circular A-130---Managing Information as a Strategic Resource
  \item Executive Order 14110---Safe, Secure, and Trustworthy Development and Use of Artificial Intelligence
  \item DoW AI Strategy, January 2026
\end{itemize}

\subsection{Research Foundation}

\begin{itemize}
  \item Dell'Acqua, F., et al.\ (2023). ``Navigating the Jagged Technological Frontier.'' Harvard Business School / BCG. Study of 758 consultants demonstrating the jagged frontier of AI capability.
  \item Brynjolfsson, E., Li, D., \& Raymond, L.\ (2025). ``Generative AI at Work.'' Stanford / MIT. Study of 5,172 customer-support agents demonstrating skill-leveling effects of AI assistance.
  \item Mollick, E.\ (2026). ``Management as AI Superpower.'' Wharton School. Research on the delegation equation and management capabilities in AI-assisted work.
  \item OpenAI (2025). ``GDPval: Measuring Economic Value of AI.'' Study of 1,320 tasks establishing expert parity benchmarks.
  \item UK Government Digital Services (2025). AI-assisted development deployment across 20,000 employees, demonstrating government-scale deployment patterns.
  \item Microsoft Work Trend Index. Study of 300,000+ employees identifying 80\% tool abandonment rate without structured training.
  \item Stanford Digital Economy Lab. Research on the apprenticeship crisis and implications for AI-assisted skill development.
\end{itemize}

% =============================================================================
% SECTION 12: APPENDICES
% =============================================================================
\section{Appendices}
\label{sec:appendices}

The following templates are used throughout the EDD development workflow. Download
each template and complete it during the appropriate phase.

\begin{longtable}{@{}p{4cm}p{4.5cm}p{5cm}@{}}
\toprule
\textbf{Template} & \textbf{Used In} & \textbf{Filename} \\
\midrule
\endfirsthead
\toprule
\textbf{Template} & \textbf{Used In} & \textbf{Filename} \\
\midrule
\endhead
\bottomrule
\endlastfoot
Problem Definition Template & Phase 1: Problem Definition & problem-definition.md \\
\midrule
QA Checklist & Phase 6: QA Review & qa-checklist.md \\
\midrule
Tool Registry Entry & Phase 7: Deployment & tool-registry-entry.md \\
\midrule
PIA Threshold Analysis & Phase 2: Compliance Review & documentation-package-outline.md \\
\midrule
Compliance Checklist & Phase 2: Compliance Review & development-journal.md \\
\end{longtable}

% ---------------------------------------------------------------------------
% APPENDIX A: Problem Definition Template
% ---------------------------------------------------------------------------
\newpage
\appendix
\section{Problem Definition Template}
\label{app:problem-definition}

\begin{longtable}{@{}p{4cm}p{9.5cm}@{}}
\toprule
\textbf{Field} & \textbf{Entry} \\
\midrule
\endfirsthead
\toprule
\textbf{Field} & \textbf{Entry} \\
\midrule
\endhead
\bottomrule
\endlastfoot
Tool Name          & \rule{9cm}{0.4pt} \\
\midrule
Developer Name     & \rule{9cm}{0.4pt} \\
\midrule
Date               & \rule{9cm}{0.4pt} \\
\midrule
Organization       & \rule{9cm}{0.4pt} \\
\midrule
Problem Statement  & Describe the problem in 2--3 sentences. What is happening now that is inefficient, error-prone, or unsustainable? \\
\midrule
Current State      & How is this task currently performed? What tools are used? How long does it take? \\
\midrule
Desired State      & What does the ideal workflow look like after the tool is deployed? \\
\midrule
Affected Users     & Who will use this tool? How many users? What are their roles? \\
\midrule
Success Criteria   & How will you know the tool works? Define 2--3 measurable outcomes. \\
\midrule
Data Requirements  & What data will the tool process? What is the classification level? Does it include PII? \\
\midrule
Deployment Platform & Where will the tool be hosted? (e.g., SharePoint, Power Platform, Azure) \\
\midrule
Estimated Timeline & How many hours/days do you expect to invest in Phases 1--7? \\
\midrule
Supervisor Approval & Name, rank, signature, date \\
\end{longtable}

% ---------------------------------------------------------------------------
% APPENDIX B: QA Review Checklist
% ---------------------------------------------------------------------------
\newpage
\section{QA Review Checklist}
\label{app:qa-checklist}

\begin{longtable}{@{}p{0.6cm}p{8cm}p{1.5cm}p{3cm}@{}}
\toprule
\textbf{\#} & \textbf{Item} & \textbf{Pass/Fail} & \textbf{Notes} \\
\midrule
\endfirsthead
\toprule
\textbf{\#} & \textbf{Item} & \textbf{Pass/Fail} & \textbf{Notes} \\
\midrule
\endhead
\bottomrule
\endlastfoot
\multicolumn{4}{l}{\textbf{Functionality}} \\
\midrule
1 & Tool performs all functions described in the Problem Definition Template & & \\
\midrule
2 & All user-facing features work without errors & & \\
\midrule
3 & Edge cases and error conditions are handled gracefully & & \\
\midrule
\multicolumn{4}{l}{\textbf{Security \& Compliance}} \\
\midrule
4 & Tool handles only unclassified data & & \\
\midrule
5 & No PII is processed without a completed PIA & & \\
\midrule
6 & Authentication and access controls are implemented & & \\
\midrule
7 & Compliance Checklist is complete and signed & & \\
\midrule
\multicolumn{4}{l}{\textbf{Documentation}} \\
\midrule
8 & Documentation package is complete & & \\
\midrule
9 & A new person could maintain the tool using only the documentation & & \\
\midrule
10 & User guide is clear and accurate & & \\
\midrule
11 & Technical documentation describes architecture and dependencies & & \\
\midrule
\multicolumn{4}{l}{\textbf{Deployment Readiness}} \\
\midrule
12 & Tool is deployed on an approved platform & & \\
\midrule
13 & Sustainment plan identifies a maintainer & & \\
\midrule
14 & Tool Registry entry is complete & & \\
\midrule
\multicolumn{4}{l}{} \\
\midrule
\multicolumn{2}{l}{\textbf{Reviewer Name / Signature:}} & \multicolumn{2}{l}{\rule{5cm}{0.4pt}} \\
\midrule
\multicolumn{2}{l}{\textbf{Date:}} & \multicolumn{2}{l}{\rule{5cm}{0.4pt}} \\
\midrule
\multicolumn{2}{l}{\textbf{Recommendation:}} & \multicolumn{2}{l}{$\square$ Approve \quad $\square$ Remediate} \\
\end{longtable}

% ---------------------------------------------------------------------------
% APPENDIX C: Tool Registry Entry
% ---------------------------------------------------------------------------
\newpage
\section{Tool Registry Entry}
\label{app:tool-registry-entry}

\begin{longtable}{@{}p{4.5cm}p{9cm}@{}}
\toprule
\textbf{Field} & \textbf{Entry} \\
\midrule
\endfirsthead
\toprule
\textbf{Field} & \textbf{Entry} \\
\midrule
\endhead
\bottomrule
\endlastfoot
Tool Name                & \rule{8.5cm}{0.4pt} \\
\midrule
Description              & Brief description of what the tool does (1--2 sentences) \\
\midrule
Developer                & Name, rank, organization \\
\midrule
Current Maintainer       & Name, rank, organization \\
\midrule
Deployment Platform      & (e.g., SharePoint, Power Platform, Azure, GitHub) \\
\midrule
URL / Location           & \rule{8.5cm}{0.4pt} \\
\midrule
Data Classification      & (e.g., Unclassified, CUI) \\
\midrule
PII Handled              & Yes / No \\
\midrule
ATO Required             & Yes / No \\
\midrule
User Count               & \rule{8.5cm}{0.4pt} \\
\midrule
Status                   & Active / Deprecated / Retired \\
\midrule
Date Deployed            & \rule{8.5cm}{0.4pt} \\
\midrule
Last Updated             & \rule{8.5cm}{0.4pt} \\
\midrule
Documentation Location   & Path or URL to the documentation package \\
\midrule
Sustainment Plan         & Who maintains it and how issues are reported \\
\end{longtable}

% ---------------------------------------------------------------------------
% APPENDIX D: PIA Threshold Analysis
% ---------------------------------------------------------------------------
\newpage
\section{PIA Threshold Analysis}
\label{app:pia-threshold}

Complete this analysis during Phase 2 (Compliance Review) for every tool. If any answer in Section~2 is ``Yes,'' consult the Privacy Officer to determine whether a full PIA is required.

\begin{longtable}{@{}p{0.6cm}p{10cm}p{2.5cm}@{}}
\toprule
\textbf{\#} & \textbf{Question} & \textbf{Yes / No} \\
\midrule
\endfirsthead
\toprule
\textbf{\#} & \textbf{Question} & \textbf{Yes / No} \\
\midrule
\endhead
\bottomrule
\endlastfoot
\multicolumn{3}{l}{\textbf{Section 1: System Description}} \\
\midrule
1 & What is the name of the tool? & \rule{2cm}{0.4pt} \\
\midrule
2 & What is the purpose of the tool? & \rule{2cm}{0.4pt} \\
\midrule
3 & Who are the intended users? & \rule{2cm}{0.4pt} \\
\midrule
\multicolumn{3}{l}{\textbf{Section 2: PII Determination}} \\
\midrule
4 & Does the tool collect, store, or display names of individuals? & \\
\midrule
5 & Does the tool collect, store, or display Social Security Numbers or DoD ID Numbers? & \\
\midrule
6 & Does the tool collect, store, or display contact information (email, phone, address)? & \\
\midrule
7 & Does the tool collect, store, or display personnel records or performance data? & \\
\midrule
8 & Does the tool collect, store, or display medical or health information? & \\
\midrule
9 & Does the tool collect, store, or display financial information? & \\
\midrule
10 & Does the tool make decisions that affect individuals (assignments, evaluations, access)? & \\
\midrule
\multicolumn{3}{l}{\textbf{Section 3: Determination}} \\
\midrule
\multicolumn{2}{l}{If all answers in Section 2 are ``No'':} & Full PIA not required \\
\midrule
\multicolumn{2}{l}{If any answer in Section 2 is ``Yes'':} & Consult Privacy Officer \\
\midrule
\multicolumn{3}{l}{} \\
\midrule
\multicolumn{2}{l}{\textbf{Analyst Name / Signature:}} & \rule{2cm}{0.4pt} \\
\midrule
\multicolumn{2}{l}{\textbf{Date:}} & \rule{2cm}{0.4pt} \\
\end{longtable}

% ---------------------------------------------------------------------------
% APPENDIX E: Compliance Checklist
% ---------------------------------------------------------------------------
\newpage
\section{Compliance Checklist}
\label{app:compliance-checklist}

Complete this checklist during Phase 2 (Compliance Review) before beginning development.

\begin{longtable}{@{}p{0.6cm}p{9cm}p{1.2cm}p{2.5cm}@{}}
\toprule
\textbf{\#} & \textbf{Item} & \textbf{Status} & \textbf{Notes} \\
\midrule
\endfirsthead
\toprule
\textbf{\#} & \textbf{Item} & \textbf{Status} & \textbf{Notes} \\
\midrule
\endhead
\bottomrule
\endlastfoot
\multicolumn{4}{l}{\textbf{Data Classification}} \\
\midrule
1 & Tool processes only unclassified data & & \\
\midrule
2 & No CUI will be processed (or platform is CUI-authorized) & & \\
\midrule
3 & No classified data will be processed at any level & & \\
\midrule
\multicolumn{4}{l}{\textbf{Privacy}} \\
\midrule
4 & PIA Threshold Analysis completed (Appendix~\ref{app:pia-threshold}) & & \\
\midrule
5 & Privacy Officer consulted (if PII is involved) & & \\
\midrule
6 & SORN requirements reviewed (if applicable) & & \\
\midrule
\multicolumn{4}{l}{\textbf{Cybersecurity}} \\
\midrule
7 & Tool will be deployed on an approved platform & & \\
\midrule
8 & AI tools used are on the approved list & & \\
\midrule
9 & ATO requirement determined (required / not required) & & \\
\midrule
10 & Cybersecurity Office consulted (if connecting to DoWIN) & & \\
\midrule
\multicolumn{4}{l}{\textbf{OPSEC}} \\
\midrule
11 & No operational details in AI prompts or code & & \\
\midrule
12 & No force structure or vulnerability data exposed & & \\
\midrule
\multicolumn{4}{l}{\textbf{Records Management}} \\
\midrule
13 & Records retention schedule identified & & \\
\midrule
14 & Documentation will be stored in an approved location & & \\
\midrule
\multicolumn{4}{l}{\textbf{Approval}} \\
\midrule
\multicolumn{2}{l}{\textbf{Developer Name / Signature:}} & \multicolumn{2}{l}{\rule{4cm}{0.4pt}} \\
\midrule
\multicolumn{2}{l}{\textbf{Supervisor Name / Signature:}} & \multicolumn{2}{l}{\rule{4cm}{0.4pt}} \\
\midrule
\multicolumn{2}{l}{\textbf{Date:}} & \multicolumn{2}{l}{\rule{4cm}{0.4pt}} \\
\end{longtable}

% =============================================================================
\end{document}
